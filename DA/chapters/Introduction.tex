
\chapter{Introduction}
\label{chapter:Introduction}


\textit{"The only truly secure system is one that is powered off, cast in a block of concrete 
and sealed in a lead-lined room with armed guards."
\textemdash  \textbf{ Gene Spafford}} \\

According to Forbes 2015~\cite{forbes:bill}, one among the topmost data breaches occurred in the 
previous years is Neiman Marcus hack.
In January 2013, many debit and credit card information of almost 350,000 customers have been
hacked. It is believed that the breach happened because of a malicious software
that was installed onto the Neiman Marcus system. This software collected all the payment
card information from customers who purchased.
This proves that the software has helped the attackers to leak all the sensitive information
through which they could get access to the system without leaving any trace of hacking.
Sensitive information can be leaked in many ways~\cite{mitre:CWE}. It can be through the environmental variables 
which contain the sensitive information about any remote server, through a log file that was
used while debugging the application where access to that file was not restricted or through a
command shell error message which indicates that the web application code has some unhandled exception.
In the last case, the attacker can take advantage of that error causing
condition in order to gain access to the system without authorisation.


The goal of this thesis is to develop an algorithm for fault localization(localizer) and also to repair the 
information exposure buggy programs. Repairing is done using the precise information like failure detection,
bug diagnosis, buggy variables which are nothing but the program variables that directly influence
the appearance of a bug in the program.
For example the buggy variable "line" reported by our information exposure checker [ref]. In order to 
repair the buggy program we make use of failure detection and bug diagnosis data to generate the quick fixes
for information exposure bugs with the help of a refactoring wizard. We have developed a localization algorithm 
in order to localize the bug. We need a localizer since the cause for the bug will not be the place where
the bug was detected but earlier in the program code where the information flows into the buggy variable.
Therefore a novel algorithm is developed in order to detect possible insert



The failure detection and bug diagnosis data is used to generate quick fixes for
buffer overflows and to support the repair process of removing the bug with a
refactoring wizard. A novel algorithm is used to detect possible insertion loca-
tions in code for the generated code patches ((a) ”in-place”—directly before the
statement which contains the bug and (b) by searching for other, not ”in-place”
locations where the bug can be fixed). Our approach for generating program
repairs is based on: code patch patterns, SMT solving and possible quick fix
locations searching in program execution paths which could affect the program
behavior by inserting a patch at a not ”in-place” location. The generated patches
are sound (e.g., do not change the behavior of the program for input which does
not trigger the bug), final (no further human refinement needed), human read-
able (no alien code), syntactically correct and compilable.
We address offline behavioral repair [24] (by modifying the source code).
Others have addressed state [8] or test-suite based program repair such as Gen-
Prog [18] and PAR [17]. The defect class which we address is inappropriate index
variable assignment which results in an incorrect usage of the buffer index range.
The fix defect class consists of input checks based on semi-defined patch pat-
terns. The aim of the quick-fix is fail-secure error mitigation (e.g., to prevent
that an attacker exploits the error in order to gain system access). The final
version of the patch is determined using SMT solving.
Program repair lies at the conjunction of two dimensions (first, an oracle is
needed to decide what is incorrect in order to detect the bug (first dimension)
and another oracle to tell what should be kept correct for sake of non-regression
(second dimension)) of software correctness [24]. We used the same SMT con-
straint system which was used to trigger the bug for defining what is incorrect
in the program. Additionally, we created a second SMTLib (constraint system
definition language used by the Z3 [7] solver) constraint system consisting of the
previously mentioned constraint system and new SMTLib constraints used to
impose input saturation constraints on the buggy variable.
Our patches are generated automatically and inserted semi-automatically offline
with the possibility to insert them also online.




\section{Information Exposure Bug}
about info expo bug.
 
 \section{Motivation with Example}
Write about motivating example.

\section{Security Hazards}
securitx hazards of info expo bugs
 
\section{Basic Terminologies}
Terminologies

\section{Contribution}
Contribution


\chapter{Technical and Scientific Fundamentals}
\label{chapter:Technical}

\section{Information Exposure Bug with examples}
about info expo bug.
 
 \section{Different Attacks}
Write about motivating example.

\section{Analysis Techniques}
like data flow and control flow

\section{Program Representation}
Terminologies

\section{Code Transformation}
Contribution

\section{Analysis Methods}
like static, dynamic or combined

\chapter{Related Work}
\label{chapter:related}