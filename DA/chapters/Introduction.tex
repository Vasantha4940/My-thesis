
\chapter{Introduction}
\label{chapter:Introduction}


\textit{"The only truly secure system is one that is powered off, cast in a block of concrete 
and sealed in a lead-lined room with armed guards."
\textemdash  \textbf{ Gene Spafford}} \\

According to the ultimate security vulnerabiltiy datasource CVE details~\cite{cve:details},
there are 2082 vulnerabilities related to information exposure. Information exposure bugs could be introduced
in different stages of software development. During design, architecture or coding phase and
can lead to exposure of critical information or can also lead to strange program behavior.
Recently, Forbes 2015~\cite{forbes:bill} published in one of the articles that
among the topmost data breaches occurred in the 
previous years Neiman Marcus hack is famous.
In January 2013, many debit and credit card information of almost 350,000 customers have been
hacked. It is believed that the breach happened because of a malicious software
that was installed onto the Neiman Marcus system. This software collected all the payment
card information from customers who purchased.
This proves that the software has helped the attackers to leak all the sensitive information
through which they could get access to the system without leaving any trace of hacking.
Sensitive information can be leaked in many ways~\cite{mitre:CWE}. It can be through the environmental variables 
which contain the sensitive information about any remote server, through a log file that was
used while debugging the application where access to that file was not restricted or through a
command shell error message which indicates that the web application code has some unhandled exception.
In the last case, the attacker can take advantage of that error causing
condition in order to gain access to the system without authorisation.


The goal of this thesis is to develop an algorithm for fault localization(localizer) and also to repair the 
information exposure buggy programs. Repairing is done using the precise information like failure detection,
bug diagnosis, buggy variables which are nothing but the program variables that directly influence
the appearance of a bug in the program.
For example the buggy variable "line" reported by the information exposure checker [ref]. In order to 
repair the buggy program, failure detection and bug diagnosis data has been used to generate the quick fixes
for information exposure bugs with the help of a refactoring wizard. We have developed a localization algorithm 
in order to localize the bug. We need a localizer since the cause for the bug will not be the place where
the bug was detected but earlier in the program code where the information flows into the buggy variable.
Therefore a novel algorithm is developed in order to detect possible insertion locations for the generated code patches.
Here the code patches can be inserted at two different locations: (i) place where the bug was found \textthreequartersemdash "in-place-fix", (ii) place where the
information flows in the buggy variable \textthreequartersemdash  "not-in-place-fix". Approach for generating
the program repair depends on the code patch patterns, SMT solving and searching possible quick fix locations searching in the 
program execution paths that will not affect the program behavior by inserting the patch at "not-in-place" location.
Generated patches do not change the program behavior for the input that doesnot trigger the bug, therefore the generated code patch
is sound. It doesnot need further human refinement(final) , no alien code(human readable), syntactically correct and compilable.
The defect class that have been addressed here is information exposure through log files , error reports and environmental variables exposure through
which sensitive information like the password or the path to remote server could be exposed and the hacker could exploit the system.
The fix defect class consists of the removal of the confidential data at the point where the information flows from the
trust boundary based on semi-defined patch patterns.The aim of the quick-fix code patch is 
information exposure error mitigation (e.g., to prevent
that an attacker exploits the error in order to gain system access or display of sensitive information).
Program repair depends on two dimensions: (i) an oracle which decides upon what is incorrect in order to detect the bug, (ii)
another oracle to decide what should be kept correct in order to attain software correctness [ref]. 
Patches are generated automatically and inserted semi-automatically offline
with the possibility to insert them also online.
90 C programs of Juliet test suite CWE-526, CWE-534, CWE-535~\cite{mitre:CWE} have been used to evaluate the developed approach. 
CWE-526 contains
information exposure through environmental variables related bugs and the potential mitigation would be to protect the 
information stored in the environmental from being exposed to the user. CWE-534 contains information 
exposure through debug log files related bugs and the potential mitigation would be to remove the 
debug files before deploying the application into produciton. CWE-535 contains information exposure through
shell error message related bugs . In all the three CWE's the common consequence of the bug is
loss of confidentiality.




\section{Information Exposure Bug}

Information Exposure Bug is an error in the software code which intentionally or unintentionally discloses sensitive information to an user who is 
not explicitly authorized to have access rights to that information.
This information could be sensitive within the developed product's own functionality like a private message, which provides information
about the product itself or which exposes the environment that is not available for the attacker and that could be very useful for the attacker like the 
installation path of that product which can be accessed remotely.
Many of the information exposures is due to many of the program errors like the PHP script error that may reveal the path of the program or
could be some timing discrepancies in encrypting the data. There are in general many kinds of problems that involve information exposures and the severity
of those problems can range vastly based on the type of the sensitive information that has been revealed by the errors.
Information exposures are also named as information leak or information disclosure.

 \section{Motivation with Example}

 
 In general detecting a information exposure bug depends on accurately finding the source code location
 where some sensitive information leaves the defined trust- boundary. This is the point where an attacker can exploit this IE vulnerability
 where sensitive information is leaked out. As already mentioned this information could be any confidential data. So in order to make sure that
 no sensitive information leaks out , a tool is developed in order to restrict the flow of information outside of the system's trust boundary at the location where the bug
 was found. In this section a real-world bug fix is presented as an example in order to depict that generating a code patch is not that easy.
 One needs an insight into the program's functionality and also the kind of bug they are dealing with. Normally, a bug can be fixed in many ways with functionally correct patches.
 Some of the generated patches may change the program behavior. So care must be take that the program behavior is not changed after the patch insertion.
 
 
 
 
 
 In this section we present two real-world bug fixes as an example to highlight the
fact that bug patch generation is not a trivial task. It needs deep insights into
the functionality of the program and merits further study. There are typically
an endless number of programs who adhere to a formal specification. As such,
a bug can be fixed with infinite number of functionally correct patches. The
automatically generated patches will change the behavior of the program or not.
We present two distinctive patches depicted in listing 1 on lines 5–6 and 11–13
with ”+” and by using an italic font. Note, that these two fixes do not change
program behavior for program input which does not trigger the bug. Listing 1
contains on line 6 code comments we present other possible quick fixes usable to
remove the buffer overflow bug located at line 12 which most likely will change
program behavior.



\begin{lstlisting}[caption={CWE-534 test programs source},label={lst:CWE534}]

if(STATIC_CONST_TRUE)
    {
        {
            char password[100] = "";
            size_t passwordLen = 0;
            HANDLE pHandle;
            char * username = "User";
            char * domain = "Domain";
            FILE * pFile = fopen("debug.txt", "a+");
            if (fgets(password, 100, stdin) == NULL)
            {
                printLine("fgets() failed");
                /* Restore NUL terminator if fgets fails */
                password[0] = '\0';
            }
            /* Remove the carriage return from the string that is inserted by fgets() */
            passwordLen = strlen(password);
            if (passwordLen > 0)
            {
                password[passwordLen-1] = '\0';
            }
            /* Use the password in LogonUser() to establish that it is "sensitive" */
            if (LogonUserA(
                        username,
                        domain,
                        password,
                        LOGON32_LOGON_NETWORK,
                        LOGON32_PROVIDER_DEFAULT,
                        &pHandle) != 0)
            {
                printLine("User logged in successfully.");
                CloseHandle(pHandle);
            }
            else
            {
                printLine("Unable to login.");
            }
            /* FLAW: Write sensitive data to the log */
      - fprintf(pFile, "User attempted access with password: %s\n", password);
      /*Source code patch(Quick fix)*/
       (*@\textbf{\textit{+ fprintf(pFile,"User attempted access with password:\n");}}@*)
            if (pFile)
            {
                fclose(pFile);
            }
        }
    }
\end{lstlisting}








\begin{lstlisting}[caption={CWE-526 test programs source},label={lst:CWE526source}]

void CWE526_bad(){
      if (staticFive == 5){
          /*FLAW:environment variable exposed*/
          - printLine(getenv("PATH"));
           /*Source code patch(Quick fix)*/
           (*@  \textbf{\textit{+ printLine(getenv(" "));}}  @*) 
           }
                 }
\end{lstlisting}
 

\begin{lstlisting}[caption={CWE-526 test programs sink},label={lst:CWE526sink}]
void printLine (const char *line){
             if(line != NULL){
                    printf("%s\n", line);
                              }
                           }
\end{lstlisting}


\section{Security Hazards}
securitx hazards of info expo bugs
 
\section{Basic Terminologies}
Terminologies

\section{Contribution}
Contribution




\chapter{Technical and Scientific Fundamentals}
\label{chapter:Technical}

\section{Information Exposure Bug with examples}
about info expo bug.
 
 \section{Different Attacks}
Write about motivating example.

\section{Analysis Techniques}
like data flow and control flow

\section{Program Representation}
Terminologies

\section{Code Transformation}
Contribution

\section{Analysis Methods}
like static, dynamic or combined

\chapter{Related Work}
\label{chapter:related}