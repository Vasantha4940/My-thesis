% Abstract for the TUM report document
% Included by MAIN.TEX


\clearemptydoublepage
\phantomsection
\addcontentsline{toc}{chapter}{Abstract}	





\vspace*{2cm}
\begin{center}
{\Large \bf Abstract}
\end{center}
\vspace{1cm}

Automatically detecting, localizing and fixing information errors is very much needed
in current generation as people are not ready to spend much on their time in debugging and
fixing the errors. In general debugging requires a lot
of time and effort. Even if the bug's root cause is known, finding the bug and fixing it is a tedious task.
The information that is exposed may be very valuable information
like passwords or any information that is used for launching many deadly attacks.
In order to fix the information exposure bug we should refactor the code in such
a way that the attacks or information exposure can be restricted. 


The main objective of this master thesis is to develop a quick fix generation tool for information
exposure bugs. Based on the available information exposure checker which detects
errors in the open source Juliet test cases: CWE-526, CWE-534 and CWE-535 a new
quick fix tool for the removal of these bugs should be developed. The bug location and
the bug fix location can be different (code lines) in a buggy program. Thus, 
there is a need for developing a bug quick fix localization algorithm based on software bug fix
localization techniques. It should help to determine the code location where the quick
fix should be inserted. Based on the bug location the developed algorithm should
indicate where the quick fix should be inserted in the program.
We can consider a bug fix to be a valid fix if it is able to remove the confidential parameter inside a
function call to a system trust boundary. After the quick fix location was determined and the format
of the quick fix was chosen then the quick fix will be inserted in the program with the
help of the Eclipse CDT/LTK API.


The effectiveness of the implemented code refactoring is checked by re-running the information flow checker
on the above mentioned open source Juliet test cases. If the checker detects no bug then
the code patches are considered to be valid. The generated patches are syntactically
correct, can be semi-automatically inserted into code and do not need
additional human refinement. The generated patches should be correct and sound.
